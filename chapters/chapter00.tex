\documentclass[../basic_graph_theory.tex]{subfiles}
\myexternaldocument{exercises}
\myexternaldocument{challenge_problems}

\begin{document}
\setcounter{chapter}{-1}
\chapter{Introduction}
\setcounter{chapter}{0} %Set chapter counter
\setcounter{section}{0}
\setcounter{equation}{0}
\setcounter{figure}{0}

These notes provide a fundamental introduction to graph theory, serving as a prerequisite for the Winter Reading Project (WRP) on Random Graphs. While it offers a solid foundation, this is not a substitute for comprehensive graph theory books. The content focuses specifically on topics essential for the WRP.

For a deeper understanding, consider referring to \textit{``Introduction to Graph Theory''} by \textit{Douglas B. West}, \textit{``Graphs and Matrices''} by \textit{R.B. Bapat}, and \textit{Benny Sudakov}'s notes available at \href{https://www2.math.ethz.ch/education/bachelor/lectures/fs2016/math/graph_theory/graph_theory_notes.pdf}{this link}.

\autoref{chap:exercises} contains simple graph theory questions to reinforce your understanding. Additionally, \autoref{chap:challenge_problems} features a list of open problems for those seeking a challenge.

While some sections cover interesting concepts beyond the scope of the Random Graphs WRP, marked by $**$, you are free to explore them at your discretion.

\end{document}