\documentclass[../basic_graph_theory.tex]{subfiles}

\begin{document}
\chapter{Introduction}
\addcontentsline{toc}{chapter}{Introduction} %Set chapter title
\setcounter{chapter}{0} %Set chapter counter
\setcounter{section}{0}
\setcounter{equation}{0}
\setcounter{figure}{0}

These notes have been prepared to serve as a basic introduction to graph theory, which is a major prerequisite for the Winter Reading Project (WRP) on Random Graphs. These notes serve as a good enough introduction to graph theory, but are in no way, an alternative to books written on graph theory. We have covered very topics, especially, only those that are absolutely required for the WRP. To learn the topic in much more depth and breadth, you can refer to \textit{``Introduction to Graph Theory''} by \textit{Douglas B. West}, \textit{``Graphs and Matrices''} by \textit{R.B. Bapat} and to \href{https://www2.math.ethz.ch/education/bachelor/lectures/fs2016/math/graph_theory/graph_theory_notes.pdf}{\textit{Benny Sudakov}'s notes}.

I have added some simple graph theory questions in Chapter , which you can try out to strengthen your concepts. Furthermore, in Chapter , I have listed some open problems, so try them at your own risk.

On another note, I have added some very interesting concepts, which will not be needed in the course of the Random Graphs WRP. You may read them up to learn more about the subject, but you are free to skip them. I have added all such portions between $**$.

\end{document}