\chapter{Colorings}

Given a graph $G$ and a positive integer $k$, a $k$-coloring is a function $K : V (G) \to \{1,...,k\}$ from the vertex set into the set of positive integers less than or equal to $k$. If we think of the latter set as a set of $k$ “colors,” then $K$ is an assignment of one color to each vertex. We say that $K$ is a proper $k$-coloring of $G$ if for every pair $u,v$ of adjacent vertices, $K(u) \neq K(v)$. If such a coloring exists for a graph $G$, we say that $G$ is $k$-colorable. Given a graph $G$, the chromatic number of $G$, denoted by $\chi(G)$, is the smallest integer $k$, such that, $G$ is k-colorable.\\
We first look at the chromatic numbers of some basic graphs.\\
$\chi(C_n)=2$ if $n=$even and $\chi(C_n)=3$ if $n=$odd\\
$\chi(K_n)=n$ and $\chi(K_{m,n})=2$\\
$\chi(P_n)=2$ if $n \ge 2$ and $\chi(P_n)=1$ if $n=1$\\

\begin{thm}
    For any graph $G$ of order $n$, $\chi(G) \le n$. 
\end{thm}
The proof of the theorem is intuitive and is left as an exercise for the reader.\\
Before moving forward, we want to point out what is famously known as the greedy algorithm. It can be used to prove several theorems related to coloring.\\
\\
\textbf{\em{Greedy algorithm}}
\begin{enumerate}
    \item[(i)] Mark all the vertices as $V=\{v_1, v_2, \dots, v_n\}$.
    \item[(ii)] Next, order the available colors in some way. We will denote them by the positive integers $1,2,\dots,n$. Then start coloring by assigning color $1$ to vertex $v_1$. Next, if $v_1$ and $v_2$ are adjacent, assign color $2$ to vertex $v_2$; otherwise, use color $1$ again. In general, to color vertex $v_i$, use the first available color that has not been used for any of $v_i$'s previously colored neighbors.
\end{enumerate}

\begin{thm}
    For any graph $G$, $\chi(G) \le \Delta(G)+1$.
\end{thm}
\begin{proof}
    Greedy algorithm uses at most $\Delta(G)+1$ colors as every vertex in the graph is adjacent to atmost $\Delta(G)$ other vertices, and hence the largest color label used is at most $\Delta(G)+1$. The equality in the theorem holds only for complete graph and odd cycles.
\end{proof}
Next, we go on to state Brooks's theorem. The proof of the theorem is significantly long and not important for the WRP. So, you are free to skip the proof.\\
\begin{thm}[Brooks's theorem]
    If $G$ is a connected graph that is neither an odd cycle nor a complete graph, then $\chi(G) \le \Delta(G)$.
\end{thm}
\begin{proof}
    
\end{proof}
The next bound involves a new concept.\\
\begin{defn}
    The clique number of a graph, denoted by $\omega(G)$, is defined as the order of the largest complete graph that is a subgraph of $G$.
\end{defn}
We will be stating the following two bounds on chromatic numbers without proof. The reader is urged to try to prove the theorems by themselves.\\
\begin{thm}
    For any graph $G$, $\chi(G) \ge \omega(G)$.
\end{thm}
\begin{thm}
    For any graph $G$ on $n$ vertices, $\frac{n}{\alpha(G)} \le \chi(G) \le n+1-\alpha(G)$.
\end{thm}
Next, we will be stating the five-color theorem.\\
\begin{thm}[Five Color theorem]
    Every planar graph is $5$-colorable.
\end{thm}
\begin{proof}
    We induct on the order of $G$. Let $G$ be a planar graph of order $n$. If $n \le 5$, then the result is clear. So suppose that $n \ge 6$ and that the result is true for all planar graphs of order $n - 1$. We know that G contains a vertex, say $v$, having $deg(v) \le 5$ (as every planar graph has $\delta(G) \le 5$).\\
    Consider the graph $G'$ obtained by removing from $G$ the vertex $v$ and all edges incident with $v$. Since the order of $G'$ is $n - 1$ (and since $G'$ is of course planar), we can apply the induction hypothesis and conclude that $G'$ is $5$-colorable. Now, we can assume that $G'$ has been colored using the five colors, named $1,2,3,4,$ and $5$. Consider now the neighbors of $v$ in $G$. As noted earlier, $v$ has at most five neighbors in $G$, and all of these neighbors are vertices in (the already colored) $G'$.\\
    If in $G'$ fewer than five colors were used to color these neighbors, then we can properly color $G$ by using the coloring for $G'$ on all vertices other than $v$, and by coloring $v$ with one of the colors that is not used on the neighbors of $v$. In doing this, we have produced a $5$-coloring for $G$.\\
    So, assume that in $G$ exactly five of the colors were used to color the neighbors of $v$. This implies that there are exactly five neighbors, call them $w_1, w_2, w_3, w_4, w_5,$ and assume without loss of generality that each wi is colored with color $i$.\\
    We wish to rearrange the colors of $G$ so that we make a color available for $v$. Consider all of the vertices of $G$ that have been colored with color $1$ or with color $3$.\\
    \em{Case 1.} Suppose that in $G$ there does not exist a path from $w_1$ to $w_3$ where all of the colors on the path are $1$ or $3$. Define a subgraph $H$ of $G$ to be the union of all paths that start at $w_1$ and that are colored with either $1$ or $3$. Note that $w_3$ is not a vertex of $H$ and that none of the neighbors of $w_3$ are in $H$. Now, interchange the colors in $H$. That is, change all of the $1$'s into $3$'s and all of the $3$'s into $1$'s. The resulting coloring of the vertices of $G$ is a proper coloring, because no problems could have possibly arisen in this interchange. We now see that $w_1$ is colored $3$, and thus color $1$ is available to use for $v$. Thus, $G$ is $5$-colorable.\\
    \em{Case 2.} Suppose that in $G$ there does exist a path from $w_1$ to $w_3$ where all of the colors on the path are $1$ or $3$. Call this path $P$. Note now that $P$ along with $v$ forms a cycle that encloses either $w_2$ or $w_4$. So there does not exist a path from $w_2$ to $w_4$ where all of the colors on the path are $2$ or $4$. Thus, the reasoning in \em{Case 1} applies. We conclude that $G$ is $5$-colorable.
\end{proof}

Following this, we now state the four-color theorem. We won't be including the proof for very obvious reasons. But rest assured it is no more a conjecture and has been proven.
\begin{thm}[Four Color theorem]
    Every planar graph is $4$-colorable.
\end{thm}